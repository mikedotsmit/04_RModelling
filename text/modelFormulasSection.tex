% Options for packages loaded elsewhere
\PassOptionsToPackage{unicode}{hyperref}
\PassOptionsToPackage{hyphens}{url}
%
\documentclass[
]{article}
\usepackage{lmodern}
\usepackage{amssymb,amsmath}
\usepackage{ifxetex,ifluatex}
\ifnum 0\ifxetex 1\fi\ifluatex 1\fi=0 % if pdftex
  \usepackage[T1]{fontenc}
  \usepackage[utf8]{inputenc}
  \usepackage{textcomp} % provide euro and other symbols
\else % if luatex or xetex
  \usepackage{unicode-math}
  \defaultfontfeatures{Scale=MatchLowercase}
  \defaultfontfeatures[\rmfamily]{Ligatures=TeX,Scale=1}
\fi
% Use upquote if available, for straight quotes in verbatim environments
\IfFileExists{upquote.sty}{\usepackage{upquote}}{}
\IfFileExists{microtype.sty}{% use microtype if available
  \usepackage[]{microtype}
  \UseMicrotypeSet[protrusion]{basicmath} % disable protrusion for tt fonts
}{}
\makeatletter
\@ifundefined{KOMAClassName}{% if non-KOMA class
  \IfFileExists{parskip.sty}{%
    \usepackage{parskip}
  }{% else
    \setlength{\parindent}{0pt}
    \setlength{\parskip}{6pt plus 2pt minus 1pt}}
}{% if KOMA class
  \KOMAoptions{parskip=half}}
\makeatother
\usepackage{xcolor}
\IfFileExists{xurl.sty}{\usepackage{xurl}}{} % add URL line breaks if available
\IfFileExists{bookmark.sty}{\usepackage{bookmark}}{\usepackage{hyperref}}
\hypersetup{
  hidelinks,
  pdfcreator={LaTeX via pandoc}}
\urlstyle{same} % disable monospaced font for URLs
\usepackage[margin=1in]{geometry}
\usepackage{graphicx,grffile}
\makeatletter
\def\maxwidth{\ifdim\Gin@nat@width>\linewidth\linewidth\else\Gin@nat@width\fi}
\def\maxheight{\ifdim\Gin@nat@height>\textheight\textheight\else\Gin@nat@height\fi}
\makeatother
% Scale images if necessary, so that they will not overflow the page
% margins by default, and it is still possible to overwrite the defaults
% using explicit options in \includegraphics[width, height, ...]{}
\setkeys{Gin}{width=\maxwidth,height=\maxheight,keepaspectratio}
% Set default figure placement to htbp
\makeatletter
\def\fps@figure{htbp}
\makeatother
\setlength{\emergencystretch}{3em} % prevent overfull lines
\providecommand{\tightlist}{%
  \setlength{\itemsep}{0pt}\setlength{\parskip}{0pt}}
\setcounter{secnumdepth}{-\maxdimen} % remove section numbering

\author{}
\date{\vspace{-2.5em}}

\begin{document}

\hypertarget{model-formulas}{%
\section{Model formulas}\label{model-formulas}}

\hypertarget{rosin-rammler-rr-model}{%
\subsubsection{Rosin Rammler (RR) Model:}\label{rosin-rammler-rr-model}}

The RR model distribution function has been used to describe the
particle size distributions of various minerals, powders and liquids of
various types and sizes. The function is particularly suited to
represent those produced by grinding, milling, and crushing operations.
The general expression of the RR model is:

\begin{equation}

  F\left(d\right) = exp\left[-(\frac{d}{l})^{m} \right]

  \label{eq : RR}
\end{equation}

where:\\
\(F(d)\) = distribution function (cum. passing)\\
\(d\) = particle size {[}mm{]}\\
Parameters \(\bar{d}\) and \(m\) are adjustable parameters
characteristic of the distribution.\\
\(\bar{d}\) = scale paramater (mean particle size {[}mm{]}) and =
\(\exp\left({-\frac{intercept}{slope}}\right)\).\\
\(m\) = slope paramater (measure of the spread of particle sizes).

The RR transformation is achieved by taking the natural log twice and
simplified as:
\[\ln\left\{-\ln\left[1-F(d)\right] \right\} = m\times \ln d - m\times \ln \bar{d} \]
Note that the RR distribution transformation is conducted on the
cummulative retained distribution (thus the \(\left[1-F(d)\right]\).)

If a distribution plots a straight line after the above RR
transformation, then the distrubution can be represented by the RR
distribution function. This will alllow percentile interpolation
according to the distribution and not a straight line between two points
obatained from screening.

\[ Y=mX + C\]

\[\underbrace{\ln\left\{-\ln\left[1-F(d)\right] \right\}}_\textrm{Y} =  \underbrace{m}_\textrm{slope}\times \underbrace{\ln d}_\textrm{X}+\underbrace{ \left(-m\times \ln \bar{d}\right)}_\textrm{C} \]

\hypertarget{back-transformation}{%
\paragraph{Back-transformation}\label{back-transformation}}

(This is really the reason why we're doing the model fitting; so that we
can better estimate the size at which a certain specified mass fraction
will pass). The back-transform is then conducted to determine the
required percentile values:

\[ x = \bar{d} \left(- \ln \left(1-Y \right)    \right)^\frac{1}{m}\]
\[size = scale parameter\times (-\ln (1-percentile))^{1/slope parameter}\]
, where the scale\_parameter; is a function of the slope and intercept
of the transf ormed RR model fit, as follows:
\[scale parameter = exp(-\frac{intercept}{slope} )\]

\begin{center}\rule{0.5\linewidth}{0.5pt}\end{center}

\hypertarget{gates-gaudin-schumann-ggs-model}{%
\subsubsection{Gates-Gaudin-Schumann (GGS)
Model:}\label{gates-gaudin-schumann-ggs-model}}

The Gates-Gaudin-Schumann plot is a graph of \textbf{cumulative \%
passing versus nominal sieve size}, with both the \(X\) and \(Y\) axes
being logarithmic plots. In this type of plot, most of the data points
(except for the two or three coarsest sizes measured) should lie nearly
in a straight line.

\[y = F(x) = \left(\frac{x}{k}\right) ^n\] or simmilarly: The above
formula can be rewritten as: \(x = k \times y ^ \frac{1}{n}\)

where:\\
\(y\) = \(F \left(x\right)\) = cumulative undersize distribution
function.\\
\(x\) = particle size,\\
\(k\) = maximum particle size of the transformed straight line
corresponding to \(100%
\) cum. passing.

Log transformation of the distribution yields:

\[\ln y = n \ln x - n \ln k\]

Applying this transformation to the measured observed distribution data
points will yield near straight lines if the data fits the model, and
interpolation on along a straight line is much easier than along the
curved arithmetic distribution function.

If the size distribution of particles from a crushing or grinding
operation does not approximate a straight line, it suggests that there
may have been a problem with the data collection, or there is something
unusual happening in the comminution process. The size modulus is a
measure of how coarse the size distribution is, and the distribution
modulus is a measure of how broad the size distribution is. Size modulus
for a size distribution can be determined from a graph by extrapolating
the straight-line portion up to 100\% passing and finding the
corresponding size value. The distribution modulus can be calculated by
choosing two points in the linear portion of the graph, calculating the
logs of the sizes and \% passing values, and calculating the slope.

RR model fits the feed streams means (F13m and F14m) better (near
straight line QQ fits) than the Oversize streams.

\end{document}
